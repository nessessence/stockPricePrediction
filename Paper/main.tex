% The original template for ICIP-2000 paper; to be used with:
%          spconf.sty  - ICASSP/ICIP LaTeX style file, and
%          IEEEbib.bst - IEEE bibliography style file.
% --------------------------------------------------------------------------
\documentclass{article}
\usepackage{spconf,amsmath,epsfig,url}

% Example definitions.
% --------------------
\def\x{{\mathbf x}}
\def\L{{\cal L}}

% Title.
% ------
\title{stockPrice Prediction}
%
% Single address.
% ---------------
\name{Student A}


\address{Department of Computer Engineering\\ Chulalongkorn Univerisy\\ Bangkok, Thailand}

% For example:
% ------------
%\address{School\\
%		 Department\\
%		 Address}
%
% Two addresses (uncomment and modify for two-address case).
% ----------------------------------------------------------

\begin{document}
%\ninept
%
\maketitle
%
\begin{abstract}
Predicting a stock price is a very challenge task.It requires high accuracy go together with high flexibility.
\end{abstract}

\section{Introduction}\label{sec:intro}

\subsection{Motivation}

Nowaday, we can see many of securities company provide a software that help customers to plan their investment strategy.
the software is also known as EA (Expert Advisor) ,this brings our team curiosity.Does the EA really work? Our target is to analyze the models and to find out which model is the most practical in the stock market.

\subsection{Previous Work}


\subsection{What We Are Going to Do}

Before our experiments, we tried some basic Deep Neural Network that base on dataset without feature engineering and found that the test score is not good as we expected.After that, our team tried to add some useful features such as moving average to the dataset but, the test score of our new model stills not good as we expected.We found that we did not deal with noises in the dataset.The noise can be generated by trading psychology or news.These leads to focusing on the noises.After that,we create the ARIMA model that relate with time-series data.The result of the model still like the others.\\We know that Newyork stock market (Dowjones) opened before Japan stock market (Nikkei) and these two stock markets are not overlapse each other.Our next milestone is using the close price of Dowjones stock market to predict the close price of Nikkei stock market.

\subsection{Organization of the Paper}

In Section Necessary Background we provide the background on the investment, basic time-series analysis and Machine Learning.



\section{Necessary Background}
\label{sec:background}


We divide topics in this part into 6 parts are as follows :\\
1. Knowledge about investment\\
2. What is time-series\\
3. ARIMA model\\
4. Feature engineering\\
5. Training a model\\
6. Backtesting


\subsection{Investment}
Everyday,there are many investors trade stocks with each other.The buyers have the money but, they would like to hold stocks instead of money , on the other hand , the sellers already have stocks but, they would like to sell their stocks and keep money instead.The market is the place that allow investors to trade with each other.Trading transaction will be happened when buyer and seller make a deal.
Price mechanism is also created by this event too.

\subsection{Basic time-series Analysis}
\begin{itemize}
\item What is time-series?\\A time series is a sequence of data that ordered by time.Stock price at time N will 
effect to stock price at time N+1
\item Stocks data is also a time series that has seasonality (each stock has opportunity day itself).
\end{itemize}


\subsection{ARIMA Model}
ARIMA, short for "Auto Regressive Integrated Moving Average", is one of the machine learning model for manipulate with time series data.The ARIMA model is divided into 3 terms :\\p is the order of AR term\\q is the order of MA term\\d is the number of differencing required to make the time series stationary.\\This model is very flexible and has strong underlying theory.Main concepts of this model are about order and differencing of the dataset. 

\subsection{Feature engineering}
Feature engineering is the method that add some useful data to the dataset.This method is created for improving the model.

\subsection{Model training}
Model training is how to make a model learns from data that feeded into it.The model will improve itself by tuning the hyperparameters to minimize loss.

\subsection{Backtesting}
Blablabla

\section{Your Proposed Method}\label{sec:yourmethod}
This section is about feature engineering and implementation's details.

\subsection{Dataset}
First, download these two dataset from link below.All dataset which is used in our experiment are from Yahoo Finance.

\begin{itemize}
\item Dowjones dataset\\
\url{https://finance.yahoo.com/quote/%5EDJI/history?p=%5EDJI}
\\Enter Time Period as you like, Choose Frequency to Daily then Press Download Data button.

\item Nikkei dataset\\
\url{https://finance.yahoo.com/quote/%5EN225/history?p=%5EN225}
\\Enter Time Period as you like, Choose Frequency to Daily then Press Download Data button.
\end{itemize}

When your download is finished, you would receive a CSV file that contains basic market data (open price, high, low, close price and volume).


\subsection{Feature engineering}
After the previous part, Our team join two CSV files together by appending Nikkei dataset at day N to Dowjones dataset at day N-1.

\subsection{Design of the model}


\subsection{Model training}

\begin{itemize}
\item 
\item This can be exactly like the paper you try to replicate. However, you should try to explain it it your own words. Most paper has space constraints and left out many details and hyperparameter choices. Describe what you use.
\end{itemize}


\section{Experimental Results}
\label{sec:exp}

Here you evaluate your work using experiments.  You start again with a very short summary, and then you
give the experimental setup. 

{\bf Note: You have to:}
\begin{itemize}
\item explain the metrics. Explain the dataset.
\item very readable, attractive plots (1 column, not 2 column plots),
proper font size
\item every plot answers a question, which you pose and extract the
answer from the plot in its discussion
\item analyze the errors. Which errors are hard and why (rare class, noisy feature)? Give example errors and why some model can handle some kind of error better than the others.
\end{itemize}

\section{Conclusions}
\label{sec:conclusion}

Here you need to summarize what you did and why this is important.  DO
NOT TAKE THE ABSTRACT and put it in the past tense. Instead, try to
highlight important results and their (potential) impact on your
problem. Say something about what you could do next and what is on
your wish list of improvements to your present method.


\section{Citing}
\label{sec:cite}

You can reference a paper by creating a key-value pair in conference.bib (at the bottom of the file, see example file). The format of the citation information follows bibtex format. Most journal websites or google scholar have a section where you can easily copy and paste the bibtex of the paper (see Figure \ref{fig:bibtex}). To cite, just use $\backslash cite\{key\}$. 

\begin{figure}
  \centering
  \centerline{\includegraphics[width=0.45\textwidth]{bibtex.png}}
  \caption{Example of how to find the bibtex on Google Scholar. Click on the quotation marks, then select bibtex.}
  \label{fig:bibtex}
\end{figure}

\section{Latex basics}
\label{sec:latex}

By this point, you probably notice that latex is quite simple and intuitive. If you are curious about more, try consulting \url{http://www.stat.pitt.edu/stoffer/freetex/latex%20basics.pdf}.

For a short version of things to know, this template already cover the basics. Note how you can reference a section, figure, or table by using $\backslash ref\{\}$, refering to some $\backslash label\{\}$ you define elsewhere in the document.

Finally, this is how you define a table.

\section{Latex editor}

There are many latex editor. One is MikTEX. There are also online tools called sharelatex \footnote{\url{https://www.sharelatex.com/}}. I recommend using this service since it require no installation. You can try for free but in order to use the more advance features (online collaboration and version control) you need to pay for its service. I have a paid subscription. If you want to use the online collaboration feature, contact me and I will host the project for your group.

\begin{table}
\centering
\begin{tabular}{|l|c|c|c|c|} % this defines how many columns and indentation of each column
 \hline % creates a horizontal line
First & 2nd & 3rd & 4th \\ %columns are separated by & and newline is \\
 \hline
 Phones & 28 & 48 & 38 \\
 Tones & n/a & n/a & n/a \\
 Graphemes & 27 & 57 & 28 \\
 Amount & 3 & 3 & 3 \\
 Wideband & yes & yes & yes \\
 Vocab & 3.7k & 7.3k & 5.4k \\
 Word count & 33k & 25k & 27k \\
 Speakers & 358 & 362 & 371 \\
 Web data amount & 38.0M & 6.4M & 16.2M \\
 \hline
\end{tabular}
\caption{Some table}
\label{tab:table}
\end{table}

% References should be produced using the bibtex program from suitable
% BiBTeX files (here: bibl_conf). The IEEEbib.bst bibliography
% style file from IEEE produces unsorted bibliography list.
% -------------------------------------------------------------------------
\bibliographystyle{IEEEbib}
\bibliography{conference}

\end{document}

